\documentclass{article}

\usepackage{polski}
\usepackage[utf8]{inputenc}
\usepackage[margin=1.5in]{geometry}
\usepackage{fancyhdr}
\usepackage{lastpage}
\usepackage{fontenc}
\usepackage{graphicx}
\usepackage{wrapfig}
\usepackage[none]{hyphenat}


\pagestyle{fancy}
\fancyhf{}
\chead{Specyfikacja implementacyjna}
\rfoot{Strona \thepage \hspace{1pt} z \pageref{LastPage}}

\title{Specyfikacja implementacyjna projektu\\ "Gra w życie"}
\author{Krzysztof Kalata, Łukasz Laskowski}

\begin{document}

\maketitle

\tableofcontents
\newpage

\section{Informacje ogólne}
\subsection{Język}
Program zostanie napisany w języku C i będzie przystosowany uruchomienia w standardzie znajdującym
się na serwerze ssh: \textit{ssh.jimp.iem.pw.edu.pl}.
\subsection{Konwencja}
Przyjmujemy, że:
\begin{itemize}
  \item nazwy modułów zaczynamy wielką literą i następne słowa (bez odstępu) również rozpoczynane są
    od wielkich liter
  \item nazwy funkcji zaczynamy małą literą i następne słowa (bez odstępu) rozpoczynane są od wielkich
    liter
  \item nazwy zmiennych zaczynamy małą literą i następne słowa (bez odstępu) rozpoczynane są również 
     małymi literami
\end{itemize}
\subsection{Wykorzystane dodatkowe biblioteki}
Poza standardowymi bibliotekami typu stdio.h czy stdlib.h, w programie wykorzystamy także biblioteki:
\begin{itemize}
  \item time.h
  \item png.h
\end{itemize}

\section{Opis modułów}
\subsection{GameOfLife}
\subsubsection{Działanie}
Jest głównym modułem, którego zadaniami są:
\begin{itemize}
  \item wywoływanie działania pozostałych modułów,
  \item interpretacja otrzymanych flag przy uruchamianiu programu,
  \item przekazywanie wydobytych z wejścia informacji do odpowiednich funkcji.
\end{itemize}
\subsubsection{Najważniejsze funkcje}
Jako że moduł ten ``prowadzi" cały program, potrzebuje on tylko jednej funkcji:\\
{\fontfamily{qcr}\selectfont main()}-- to w tej funkcji sprawdzane są
flagi, otwierane są uchwyty do plików, sprawdzenie, czy zostało to przeprowadzone bez problemu
oraz przekazanie uchwytu do pliku wejściowego do \textit{Readera}.\\
Funkcja następnie wywołuje działanie głównego modułu, przeprowadzającego generacje, a~po każdym
wykonanym kroku (w przypadku flagi -sbs jedynie po wybranych) uruchamia funkcje
konwertujące wyniki do odpowiednich rozszerzeń oraz plików.

\subsection{Reader}
\subsubsection{Działanie}
Moduł czytający dane, którego rolę można opisać następująco:
\begin{itemize}
  \item analizuje plik wejściowy
  \item odczytuje dane z pliku wejściowego
  \item tworzy struktury potrzebne do przechowywania siatki w pamięci
  \item uzupełnia nowo stworzoną planszę informacjami z pliku wejściowego
\end{itemize}
\subsubsection{Najważniejsze funkcje}
Moduł ten również posiada dwie funkcje:\\
{\fontfamily{qcr}\selectfont readFile()}-- po otrzymaniu ścieżki do pliku wejściowego, analizuje
poprawność danych w~pliku oraz wpisuje je do utworzonej struktury.\\
{\fontfamily{qcr}\selectfont checkFile()}-- funkcja ta otrzymuje ścieżkę do pliku wejściowego
jeszcze przed funkcją readFile(), aby sprawdzić poprawność danych. Sprawdzane jest przede wszystkim
to, czy~w~pliku podane są wymiary planszy (jako dwie pierwsze liczby) oraz czy ciąg opisujący
plaszę składa się wyłącznie z zer i jedynek.

\subsection{Simulation}
\subsubsection{Działanie}
Jest to serce \textit{Gry w życie}. To w tym module następują takie czynności jak:
\begin{itemize}
  \item pobranie od readera planszy wraz z informacją o liczbie iteracji do wykonania
  \item przygotowuje kopię planszy, niezbędną do prostego przeprowadzenia symulacji
  \item tworzy odpowiednią ilość generacji
  \item komunikuje się w modułem eksportującym wyniki każdej iteracji do plików png i txt
\end{itemize}
W module tym operujemy na samej planszy, zatem to też tu pojawi się definicja struktury
spełniającej rolę swego rodzaju pojemnika, zawierającego wszystkie niezbędne informacje
o~aktualnej siatce:\\
{\fontfamily{qcr}\selectfont
struct b\{ \\
int height, width;\\
int **life, **copy;\\
\} *board;}

\subsubsection{Najważniejsze funkcje}
{\fontfamily{qcr}\selectfont generate()}-- funkcja ta aktualizuje stany wszystkich komórek
na planszy\\
{\fontfamily{qcr}\selectfont copy()}-- funkcja przenosząca dane z wektora \textit{life} na
wektor \textit{copy}\\

\subsection{Converter}
\subsubsection{Działanie}
Jest to moduł tworzący obrazy png oraz odpowiadające im pliki txt na podstawie planszy otrzymanej
z przeprowadzonych symulacji. Pliki txt tworzone są po to, by móc uruchomić program od wybranej
iteracji, bez konieczności tworzenia oddzielnego testu.
\subsubsection{Najważniejsze funkcje}
{\fontfamily{qcr}\selectfont toPNG()}-- funkcja eksportująca planszę do obrazu formatu PNG.\\
{\fontfamily{qcr}\selectfont toTxt()}-- funkcja eksportująca planszę do formatu txt.

\subsection{Generator}
Jest to oddzielny program, który w linii wywołania otrzymuje dwie wartości -- wysokość i szerokość
planszy. Na ich podstawie tworzy plik tekstowy w formacie odpowiadającym plikom wejściowym
\textit{Gry w życie}, uzupełniając go losowo symbolami `1' i `0'.

\subsection{Diagram modułów}
\begin{center}
  \includegraphics[width=6in,height=5in,keepaspectratio]{diagram.png}
\end{center}

\newpage
\section{Testowanie}
\subsection{Konwencje}
Testy będą przebiegały w kilku fazach:
\begin{itemize}
  \item testowanie poprawnego współdziałania modułów, poprzez dostarczenie im wygenerowanych
    losowo w Generatorze plansz i obserwowanie w trybie \textit{steb-by-step} wczytywania,
    poprawności zapisu, przekazywania struktur itp.
  \item testowanie ogólnego działania algorytmów modułu Simulation. W tej części testy będą odbywać
    się na odgórnie zdefiniowanych planszach z wbudowanymi strukturami typu \textit{Niezmiennych}
    (klocek, łódź, bochenek), \textit{Oscylatorów} (żabka, krokodyl), czy \textit{Statków} (Glider,
    Dakota) i obserwowanie, czy zachowują się poprawnie.
  \item wypisywanie plansz na konsolę lub do plików tekstowych w celu sprawdzenia poprawności
    przeprowadzanych iteracji
\end{itemize}
\subsection{Użyte narzędzia}
Poza testowaniem poprawności działania algorytmów, przetestujemy program pod względem poprawnego
zarządzania pamięcią. Użyjemy do tego dwóch narzędzi:
\begin{itemize}
  \item program Valgrind na platformie Linux
  \item program Dr. Memory na platformie Windows
\end{itemize}

\end{document}
