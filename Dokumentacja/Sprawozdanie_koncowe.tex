\documentclass{article}

\usepackage{polski}
\usepackage[utf8]{inputenc}
\usepackage[margin=1.5in]{geometry}
\usepackage{fancyhdr}
\usepackage{lastpage}
\usepackage[none]{hyphenat}
\usepackage{graphicx}
\usepackage{wrapfig}
\usepackage{fontenc}

\pagestyle{fancy}
\fancyhf{}
\chead{Sprawozdanie końcowe - \textit{Gra w życie}}
\rfoot{Strona \thepage \hspace{1pt} z \pageref{LastPage}}

\title{Sprawozdanie końcowe z projektu\\automatu komórkowego\\``Game of life''}
\author{Krzysztof Kalata, Łukasz Laskowski}

\begin{document}

\maketitle
\newpage

\tableofcontents
\newpage

\section{Opis ogólny}
\subsection{Nazwa programu}
Program nazywa się ``gameOfLife''.

\subsection{Poruszany problem}
Celem projektu było napisanie programu pozwalającego zasymulować \textit{Grę w
życie}. \\Jest to przykład automatu komórkowego symulującego na swój sposób
życie komórek. Każda~komórka znajdować się może w jednym z dwóch stanów --
może być żywa lub martwa. \\Jako że umieszczone są w prostokątnym układzie
współrzędnych, każda komórka niepołożona przy brzegu posiada dokładnie 8
sąsiadów. Komórka żywa pozostaje, taką wyłącznie posiadając 2 lub 3 żywe komórki.
W każdym innym przypadku komórka umiera z samotności lub zatłoczenia. Jeśli
komórka była martwa, ale posiadała dokładnie trzech żywych sąsiadów w~jednej iteracji,
w kolejnej jej miejsce zajmuje komórka nowo narodzona. Ważną
kwestią jest także to, że sąsiedztwo można definiować na dwa sposoby – wliczając
wszystkie 8 komórek (Moore’a) lub licząc tylko te bezpośrednio stykające się
krawędziami z obserwowaną komórką (von Neumanna).


\subsection{Przykłady uruchomienia}
Poniżej podane są przykładowe uruchomienia programu, a następnie opis
tego, co nastąpi po ich wywołaniu.

\begin{itemize}
  \item ./gameOfLife in.txt /first -n 100 -s 1\\
  Program odczyta plik wejściowy in.txt z katalogu z grą, wykona 100 generacji w~trybie
step-by-step z których wybrane przez użytkownika zapisze w katalogu o
nazwie \textit{first} (katalog musi istnieć) sto par plików .txt i .png ponumerowanych
kolejno (0001, 0002, 0003, 0004,0005…), chyba że wcześniej program osiągnie stan
ustalony, wtedy przerwie generowanie kolejnych.
  \item ./gameOfLife in.txt /first -n 100 \\
  Program odczyta plik wejściowy \textit{in.txt} z katalogu z grą, wykona 100 generacji,
które~zapisze w katalogu o nazwie \textit{first} (katalog musi istnieć) sto par plików .txt i
.png ponumerowanych kolejno (0001, 0002, 0003, 0004, 0005…) chyba że
wcześniej program osiągnie stan ustalony, wtedy przerwie generowanie
kolejnych.
  \item ./gameOfLife in.txt /first \\
  Program odczyta plik wejściowy \textit{in.txt} z katalogu z grą, wykona 50 (wartość
podstawowa) generacji, które zapisze w katalogu o nazwie \textit{first} (katalog musi
istnieć) pięćdziesiąt par plików .txt i .png ponumerowanych kolejno
(0001,0002,0003,0004,0005…), chyba że wcześniej program osiągnie stan
ustalony, wtedy przerwie generowanie kolejnych.
  \item ./gameOfLife in.txt \\
  Symulacja się nie wykona, a program wyświetli jedynie instrukcję uruchamiania.
  \item ./gameOfLife /first \\
  Symulacja się nie wykona, a program wyświetli jedynie instrukcję uruchamiania.
  \item ./gameOfLife in \\
  Symulacja się nie wykona, a program wyświetli jedynie instrukcję uruchamiania.

\end{itemize}

\section{Osiągnięte założenia i zmiany}
\subsection{GameOfLife}
\subsubsection{Założenia}
Moduł ten jest trzonem programu, kieruje on jego działaniem od początku, aż
do samego końca. Na początku analizuje otrzymane flagi, zwraca
odpowiednie komunikaty błędu lub kolejno uruchamia wczytywanie pliku,
przeprowadzenie symulacji i na końcu zwalnia zajętą pamięć.

\subsubsection{Zmiany względem specyfikacji}
Teraz to nie ten moduł uruchamia funkcje konwertujące wyniki do plików
wyjściowych w formatach .png i .txt. W końcowej wersji działanie to zostało
przesunięte do modułu \textit{Simulation}.

\subsection{Reader}
\subsubsection{Założenia}
Ta część programu po otrzymaniu ścieżki pliku wejściowego sprawdza
poprawność danych, rezerwuje pamięć pod strukturę, jak i dwie tablice (life i
copy) o zadanych wymiarach. Następnie wypełnia je danymi pobranymi z
pliku.

\subsubsection{Zmiany względem specyfikacji}
Brak zmian modułu w stosunku do specyfikacji.

\subsection{Simulation}
\subsubsection{Założenia}
To tu wykonywana jest największa część pracy programu. Moduł ten
wykonuje operacje na~planszy gry, analizuje jej stan, na jego podstawie
tworzy kolejne generacje na kopii planszy, powtarza to zadaną liczbę razy, w
międzyczasie sprawdzając, czy nie nastąpił stan ustalony. Dodatkowo obsługuje
moduł step-by-step (sbs) oraz wywołuje funkcje konwersji do plików
wyjściowych o rozszerzeniach .txt i .png.

\subsubsection{Zmiany względem specyfikacji}
Względem specyfikacji działanie modułu zostało rozszerzone o obsługę
modułu step-by-step oraz o wywoływanie funkcji konwersji do plików
wyjściowych.

\subsection{Converter}
\subsubsection{Założenia}
Moduł ten składa się z dwóch funkcji, które wywołuje \textit{Simulation}. Jedna z
nich zajmuje się konwersją do pliku .txt. Tworzy jego nazwę oraz wypełnia go
danymi, a następnie zapisuje go w określonym miejscu. Druga natomiast
analogicznie tworzy i zapisuje plik .png.

\subsubsection{Zmiany względem specyfikacji}
Brak zmian w stosunku do specyfikacji.

\subsection{Generator}
\subsubsection{Założenia}
Jest to oddzielny program, który w linii wywołania otrzymuje dwie wartości –
wysokość i szerokość planszy. Na ich podstawie tworzy plik tekstowy w
formacie odpowiadającym plikom wejściowym Gry w życie, uzupełniając go
losowo symbolami ‘1’ i ‘0’.

\subsubsection{Zmiany względem specyfikacji}
Brak zmian w stosunku do specyfikacji.

\subsection{Pozostałe zmiany}
W programie musieliśmy dodać także biblioteki, których nie przewidzieliśmy podczas pisania specyfikacji. Są to biblioteki:
\begin{itemize}
    \item string.h -- do podstawowych działań na ``stringach'' w C (strcpy, strcat, strlen)
    \item getopt.h -- do obsługi załączanych flag (getopt)
\end{itemize}
Nie musieliśmy natomiast używać biblioteki time.h, ponieważ zmieniliśmy metodę tworzenia nazw plików wyjściowych na zwykłe numerowanie (0001, 0002...), zamiast godziny i daty, ponieważ problem stanowiłaby sytuacja, gdy kilka plików wygenerowałoby się w jednej sekundzie.\\
Zmianom uległ również lekko diagram modułów, a aktualny prezentuje się następująco:
\begin{center}
  \includegraphics[width=6in,height=5in,keepaspectratio]{diagram2.png}
\end{center}

\section{Opis załączonych plików}
\subsection{Moduły programów}
Wszystkie pliki składające się na modułu programu zz wyjątkiem Generatora
znajdują się w katalogu głównym w dwóch wersjach (.h oraz .c). Generator
znajduję się w podkatalogu \textit{/in} jedynie w wersji .c.

\subsection{Dokumentacja}
Dokumentacja funkcjonalna, jak i implementacyjna znajdują się w podkatalogu
\textit{/Dokumentacja}, każda w dwóch wersjach (.pdf oraz .tex).

\subsection{Testy}
Przykładowe testowe plansze znajdują się w podkatalogu \textit{./in/testy struktur},
gdzie przechowywane są w formacie .txt. Natomiast testy poszczególnych
modułów znajdują się w katalogu głównym i ich nazwy zaczynają się od
\textit{test\_}.


\end{document}
